\documentclass{resume} % Use the custom resume.cls style

\usepackage[left=0.4 in,top=0.4in,right=0.4 in,bottom=0.4in]{geometry} % Document margins
\newcommand{\tab}[1]{\hspace{.2667\textwidth}\rlap{#1}} 
\newcommand{\itab}[1]{\hspace{0em}\rlap{#1}}
\name{Riccardo Periotto} % Your name

\address{\href{https://www.linkedin.com/in/riccardoperiotto}{\faLinkedinSquare \space riccardoperiotto} \quad \href{mailto:riccardo.periotto@gmail.com}{\faEnvelope \space riccardo.periotto@gmail.com}\quad \href{tel:+393662570766}{+39 3662570766} \quad \href{https://www.github.com/riccardoperiotto}{\faGithubSquare \space riccardoperiotto}}

\begin{document}

%----------------------------------------------------------------------------------------
%	SUMMARY
%----------------------------------------------------------------------------------------

\begin{rSection}{SUMMARY}

{Software Engineer in robotics, combining strong foundations in Computer Science and Autonomous Systems with hands-on experience. Currently applying my skills to real-world robotics challenges in industrial automation.}

\end{rSection}

%----------------------------------------------------------------------------------------
% TECHINICAL SKILLS SECTION	
%----------------------------------------------------------------------------------------
\begin{rSection}{SKILLS}

% between ROS and robotics [\mbox{Robot Programming}, \mbox{Industrial Robotics}, \mbox{Mobile Robotics}, ], eventually
\begin{tabular}{ @{} >{\bfseries}l @{\hspace{3ex}} p{16.1cm} }
Programming & C, C++, C\#, Python, TypeScript, Java, R, \mbox{Algorithms \& Data Structures}, OOP, \mbox{.NET Framework}, Django, React.js, Vue.js, \mbox{Docker \& Microservices}, DDS, gRPC, MQTT, Unity, Git, CI/CD, HTML5,  CSS, \mbox{SQL \& NoSQL} \\
Mechatronics & \mbox{ROS 2}, Robotics, \mbox{Automatic Control}, \mbox{Dynamic Modeling}, \mbox{Motion Planning}, \mbox{Sensor Fusion}, SLAM, Gazebo, ArduPilot, \mbox{Industry 4.0} \\
AI \& Modeling & JAX, PyTorch, TensorFlow, \mbox{Machine Learning}, \mbox{Deep Learning}, \mbox{Reinforcement Learning}, \mbox{Unsupervised Learning}, \mbox{Probabilistic Methods}, \mbox{Bayesian Networks}, MATLAB, Simulink, Maple
\end{tabular}\\

\end{rSection}

%----------------------------------------------------------------------------------------
%	WORK EXPERIENCE SECTION (relevant for the role)
%----------------------------------------------------------------------------------------

\begin{rSection}{EXPERIENCE}

% Given example of experience entry:
% \textbf{Role Name} \hfill Jan 2017 - Jan 2019\\
% Company Name \hfill \textit{San Francisco, CA}
%  \begin{itemize}
%     \itemsep -3pt {} 
%      \item Achieved X\% growth for XYZ using A, B, and C skills.
%      \item Led XYZ which led to X\% of improvement in ABC
%     \item Developed XYZ that did A, B, and C using X, Y, and Z. 
%  \end{itemize}

\textbf{Software Engineer for Robotics} \hfill Dec 2023 -- Present\\
Agile Robots SE \hfill \textit{Munich, Germany}
\begin{itemize}
    \itemsep -3pt {}
    \item Developed backend and frontend features for \href{https://www.agile-robots.com/en/solutions/agilecore/}{AgileCore}, enabling faster deployment of automation solutions.
    \item After $\sim$6 months, contributed to a \textbf{full product rewrite}, making key architectural decisions on communication protocols and interfaces, data persistence, database design, log management, user authentication and more.    
    \item Collaborated closely with system integrators and application engineers, including \textbf{on-site validation in China}, to align software with real-world operational requirements.
    \item Designed and implemented machine tending applications for industrial robots, consistently meeting \textbf{production KPIs} (e.g., CNC exchange in $\leq$30\,s) by optimizing robot parameters, motion paths, and speed profiles.
    \item Contributed to an external cross-department project based on a new C\#/.NET stack with PLC integration, demonstrating \textbf{rapid learning and new technologies adaptability}.
    \item Led the full development of a Vue.js HMI for camera calibration, enabling users to efficiently collect image data and run calibration algorithms, \textbf{from requirements gathering to UX design and implementation}.
\end{itemize}

\textbf{Research Intern} \hfill Jan 2023 - Jul 2023\\
Ericsson \hfill \textit{Stockholm, Sweden}
\begin{itemize}
    \itemsep -3pt {} 
    \item Designed an optimization-based control algorithm for teleoperation, accounting for network delays and physical obstacles. Validation in simulation and hardware with 100\% adherence to safety constraints.
    % \item Designed a control algorithm for teleoperation, accounting for physical obstacles and network delays.
    \item Built the project as the basis for my master’s thesis, resulting in a first-author paper at ECC. \href{https://kth.diva-portal.org/smash/record.jsf?pid=diva2%3A1846846&dswid=1211}{Thesis}, \href{https://ieeexplore.ieee.org/document/10590767}{Paper}
\end{itemize}

\textbf{Software Engineer} \hfill Jan 2021 - Sep 2021\\
BM Group Polytec S.p.A. \hfill \textit{Trentino, Italy}
\begin{itemize}
    \itemsep -3pt {} 
    \item Worked on multiple projects in the company’s software department, ranging from internal tools to cloud platforms and industrial automation systems.
    \item Implemented real-time data streaming from multiple IoT sensors for hydroelectric plants, handling high-frequency logging, storage, and visualization on dashboards; automated daily PDF reports for different stakeholders, improving operational monitoring and decision-making.
    \item Developed and deployed software for industrial robotics applications for customers such as Acciaierie d'Italia and Iperceramica, including on-site deployment and tuning to meet operational KPIs. See \href{https://youtu.be/XaXRUYPYLRw}{video} to learn more.
\end{itemize}

\newpage


% Not relevant anymore
% \textbf{Computer Science Intern} \hfill Oct 2019 - Feb 2020\\
% SpazioDati S.r.l. \hfill \textit{Trento, Italy}
% \begin{itemize}
%     \itemsep -3pt {} 
%     \item Developed a web application analyzing company data as part of my bachelor’s degree project.
% \end{itemize}

\end{rSection} 

%----------------------------------------------------------------------------------------
%	EDUCATION SECTION
%----------------------------------------------------------------------------------------

\begin{rSection}{Education}

% Given example of education entry:
% {\bf Master of Computer Science}, Stanford University \hfill {Expected 2020 | 2014 - 2017}\\
% Relevant Coursework: A, B, C, and D.

{\bf MSc in Autonomous Systems}, EIT Digital \hfill {2023 - 2024}\\
First year: University of Trento (UniTN), Second year: Royal Institute of Technology (KTH) \\
Focus on robotics and AI, with additional studies in entrepreneurship and business.

{\bf BSc in Computer Science}, University of Trento \hfill {Sep 2017 - Jul 2020} \\
Main subjects: Algorithms, Software Engineering  \quad Grade: 110/110 with honours

{\bf High School Diploma in Computer Science}, ITT Marconi Rovereto \hfill {Sep 2014 - Jul 2017} \\
Main subjects: Programming, Network Systems \quad Grade: 100/100 with honours

\end{rSection}

\begin{rSection}{Additional Information}

\textbf{Projects:} I share some of the main projects I built or contributed to on \href{https://github.com/riccardoperiotto}{GitHub}. Feel free to check them out.

\textbf{Languages:} Italian (native), English (full professional proficiency), German (intermediate).

% personal touch?
% \textbf{Other:} Full driving license (Category B) and own car. I come from a tourist village (see \href{https://maps.app.goo.gl/KioJq2S5FDpbuEKF6}{Maps}) and have always done seasonal jobs. Passionate about sports (running, climbing, snowboarding, and recently surfing).

\end{rSection}

\end{document}
